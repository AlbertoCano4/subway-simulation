\chapter{Improvements for the next project}

\noindent\textbf{Areas for Improvement and Critical Assessment}
~\\[0,5cm]

While the current simulation successfully models a metro system using discrete-event principles and presents a solid foundational structure, future developments could significantly enhance its scalability, realism, and analytical depth—particularly through the integration of Artificial Intelligence (AI) and fuzzy logic techniques.
~\\[0,5cm]
Although the project is already organized into modular components , further abstraction would make the architecture more extensible. Defining higher-level classes, such as a MetroSystem controller or abstract entities like TransportUnit, would allow for easier implementation of features like multiple metro lines, interchange stations, or AI-driven routing and scheduling.
~\\[0,5cm]
Another technical improvement involves the modeling of travel times. Currently, all stations are treated as equidistant, which limits simulation realism. Introducing a matrix of variable distances or travel times between stations would allow for more accurate timing logic and could serve as a basis for algorithms that optimize train frequency based on real-time conditions.
~\\[0,5cm]
In terms of AI, future versions of the project could implement machine learning algorithms to dynamically adjust train frequency based on historical passenger data or predicted demand patterns. Supervised models could forecast peak times or station-level congestion, while reinforcement learning agents could be trained to manage scheduling decisions in real time to minimize waiting time and maximize system efficiency.
~\\[0,5cm]
Additionally, fuzzy logic could be incorporated to model more human-like or uncertain behaviors, such as passengers choosing between two stations or trains based on subjective criteria (e.g., "the train is too full", "I might wait for the next one"). This would bring more nuance and realism to passenger simulation, especially when decision-making cannot be captured by binary logic.
~\\[0,5cm]
Another key area for improvement is visualization. The project currently logs events to a CSV file for analysis, but lacks a visual interface. Adding a basic UI or data dashboard using tools like matplotlib, Plotly, or NetworkX would not only improve presentation quality but also help validate the model visually and make it more engaging for stakeholders.
~\\[0,5cm]
From a data perspective, the simulation currently uses idealized inputs. Integrating real-world datasets—such as passenger volumes, demographic trends, or transport usage statistics—would boost the system’s predictive value. Real station names, arrival behaviors, and urban constraints could be used to reflect the conditions of the Madrid Metro more accurately.
~\\[0,5cm]
Finally, the existing logging system could be extended to automatically analyze performance metrics, such as average waiting times, train occupancy, and bottlenecks. With AI and fuzzy logic integrated, the system could evolve into a smart decision-support platform for transport planners.
~\\[0,5cm]
In conclusion, this project provides a robust technical foundation and a clear social purpose. With the planned integration of AI, fuzzy logic, and real-world data, it has the potential to evolve into a sophisticated simulation and optimization tool for designing more intelligent, adaptable, and inclusive public transportation systems.
Lastly, from a user interaction perspective, the system lacks a visual interface that would aid in understanding the model and its results. A graphical interface, or at least a visual representation of the network and proposed routes, would have significantly increased the project's technical value and presentation quality.
~\\[0,5cm]
In summary, while this project offers a solid foundation, it also highlights critical limitations. These shortcomings must be addressed in future developments if simulation is to be a useful and interactive tool for urban transport planning.


