
\chapter*{\center \Large  Code explanation }

This project simulates a metro system using SimPy, a discrete-event simulation library in Python, alongside NumPy for efficient data handling and random number generation. The system models a circular metro line with stations, trains, and passengers, where each component interacts dynamically according to time-dependent demand patterns. All major events (such as passenger arrivals, boardings, alightings, and train operations)are logged into a CSV file with precise timestamps. The number of active trains is automatically adjusted throughout the day based on predefined frequency schedules that adapt to peak and off-peak hours, ensuring realistic and scalable system behavior.


~\\[0,5cm]
\noindent\textbf{The main script:} central coordinator of the entire system. It initializes the SimPy environment, sets up time-handling utilities, and configures the logging of simulation events into a CSV file. In this file, relevant details like event type, station, time, and train identifiers are recorded. The script defines passenger arrival rates, generates stations along a metro line, and dynamically creates trains throughout the simulation. It adjusts train frequency according to simulated demand based on the time of day, ensuring the number of active trains reflects real-world peaks and valleys in usage.
~\\[0,5cm]
\noindent\textbf{The line module:} encapsulates the structure of the metro line itself. It manages the ordered list of stations, determines the next station a train should move to based on its direction, and calculates the number of stops between any two stations. It also includes a method to verify whether a train is heading in the correct direction to serve a passenger. This module acts as the network logic layer, ensuring passengers and trains follow consistent, rule-based movement through the circular metro system.

~\\[0,5cm]
\noindent\textbf{The Station module:} is responsible for simulating individual Station objects. Each station includes its name, current waiting passengers, and a rate at which new passengers are generated. This arrival rate changes dynamically depending on the time, simulating rush hours and off-peak periods. New passengers are assigned random destinations different from their current station, and all stations are connected to the metro line for consistent routing. This module helps simulate demand variability and station-level dynamics.

~\\[0,5cm]
\noindent\textbf{The Passenger module:} contains the Passenger class, which is a lightweight representation of users in the system. Each passenger holds an identifier and a destination station. The class includes a method for string formatting, useful for displaying or logging purposes. While simple, this module plays a crucial role in simulating realistic and individualized passenger movement throughout the network.


~\\[0,5cm]
\noindent\textbf{The Train module:} defines the logic for each Train object in the simulation. Each train has attributes such as its direction, capacity, current position, onboard passengers, and a reference to the overall line. Trains cycle through stations, allowing passengers to alight at their destination and boarding new ones when heading in the correct direction. The class includes mechanisms for avoiding station collisions and varying train travel time based on the time of day. It is also linked to the event logger, ensuring every action is recorded for later analysis.
~\\[1cm]
The main simulation function runs the train generation process, ensuring that the number of trains in circulation dynamically adjusts based on demand and the time of day. It also closes the event log file at the end of the simulation.
~\\[0,5cm]

In conclusion, the current version of the code is highly modular, with each file contributing a specific piece of the larger simulation. The design is scalable, easy to maintain, and well-prepared for integrating intelligent systems such as AI-based scheduling, passenger demand forecasting, or adaptive control strategies




